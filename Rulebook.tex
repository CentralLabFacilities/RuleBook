%%%%%%%%%%%%%%%%%%%%%%%%%%%%%%%%%%%%%%%%%%%%%%%%%%%%%%%%%%%%%%%%%%%%%%%%%%%%%%%
%%
%%          $Id: Rulebook.tex 2014-12-12 balkce $
%%    author(s): RoboCupAtHome Technical Committee(s)
%%  description: introduction to RoboCupAtHome
%%
%%%%%%%%%%%%%%%%%%%%%%%%%%%%%%%%%%%%%%%%%%%%%%%%%%%%%%%%%%%%%%%%%%%%%%%%%%%%%%%
\documentclass[11pt, twoside, openright, a4paper, chapterprefix]{scrbook}
\usepackage[inner=2.5cm, outer=2.5cm, top=4cm, bottom=4cm]{geometry}

%%% PACKAGES %%%%%%%%%%%%%%%%%%%%%%%%%%%%%%%%%%%%%%%%%%%%%%%%%%%%%%%%%%%%%%%%%%
\input{./setup/packages.tex}
\usepackage[titletoc]{appendix}
\usepackage{enumitem}
\usepackage{mathtools}
\usepackage{gensymb}
\setlist{noitemsep}

%%% SubfigureSetup %%%%%%%%%%%%%%%%%%%%%%%%%%%%%%%%%%%%%%%%%%%%%%%%%%%%%%%%%%%%
%\renewcommand{\subfigtopskip}{5pt}        % default is 10pt
%\renewcommand{\subfigbottomskip}{5pt}     % default is 10pt
%\renewcommand{\subfigcapskip}{3pt}        % default is 10pt
%\renewcommand{\subfigcapmargin}{7pt}      % default is 10pt

%%% TweakList-Setup %%%%%%%%%%%%%%%%%%%%%%%%%%%%%%%%%%%%%%%%%%%%%%%%%%%%%%%%%%%
\renewcommand{\itemhook}{%                 % modify itemize-spacing
  \setlength{\topsep}{2pt}%
  \setlength{\partopsep}{1pt}%
  \setlength{\itemsep}{-1pt}%
}
\renewcommand{\enumhook}{%                 % modify enumerate-spacing
  \setlength{\topsep}{2pt}%
  \setlength{\partopsep}{1pt}%
  \setlength{\itemsep}{-1pt}%
}
\renewcommand{\descripthook}{%             % modify description-spacing
  \setlength{\topsep}{2pt}%
  \setlength{\partopsep}{1pt}%
  \setlength{\itemsep}{-1pt}%
}

\setkomafont{title}{\normalfont}
\setkomafont{sectioning}{\normalfont\bfseries}
\addtokomafont{caption}{\small}
\setkomafont{captionlabel}{\small\bfseries}
\setkomafont{descriptionlabel}{\normalfont\bfseries}
\renewcommand*{\chapterformat}{\LARGE{Chapter \thechapter}}

%%% MACROS %%%%%%%%%%%%%%%%%%%%%%%%%%%%%%%%%%%%%%%%%%%%%%%%%%%%%%%%%%%%%%%%%%%%
\input{./setup/active_version.tex}
\graphicspath{{\YEAR/}{./images/}}
\input{./setup/macros.tex}
\input{./setup/abbrevix.tex}



\makeindex                                % generate index
\makeabbex                                % generate abbreviations

%%% DOCUMENTINFO %%%%%%%%%%%%%%%%%%%%%%%%%%%%%%%%%%%%%%%%%%%%%%%%%%%%%%%%%%%%%%
\hypersetup{
  pdftitle     = {RoboCup@Home Rules and Regulations},
  pdfsubject   = {RoboCup@Home Rulebook},
  pdfauthor    = {RoboCup@Home Technical Committee},
  pdfkeywords  = {RoboCup, @Home, Rules, Competition},
  colorlinks   = true,
  anchorcolor  = blue,
  linkcolor    = blue,
  urlcolor     = blue, 
}

%%% HEADINGS & PAGE STYLE %%%%%%%%%%%%%%%%%%%%%%%%%%%%%%%%%%%%%%%%%%%%%%%%%%%%%
\newcommand{\footline}{RoboCup@Home Rulebook / \rulebookVersion}
\pagestyle{fancy}
\renewcommand{\chaptermark}[1]{\markboth{\chaptername\ \thechapter. \ #1}{}}
\renewcommand{\sectionmark}[1]{\markright{\thesection \ #1}{}\renewcommand{\currentTest}{#1}}
\fancyhf{}
\fancyhead[LE,RO]{\thepage}
\fancyhead[RE]{\sffamily\rightmark}
\fancyhead[LO]{\sffamily\leftmark}
\fancyfoot[C]{\scriptsize \sffamily \footline{}}
\fancypagestyle{plain}{
        \fancyhf{}
        \fancyhead[LE,RO]{\thepage}
        \fancyhead[RE]{\sffamily\rightmark}
        \fancyhead[LO]{\sffamily\leftmark}
        \fancyfoot[C]{\scriptsize \sffamily \footline{}}
		\renewcommand{\headrulewidth}{0.5 pt}
}
\fancypagestyle{empty}{
        \fancyhf{}
        \fancyhead{}
        \fancyfoot[C]{\scriptsize \sffamily \footline{}}
		\renewcommand{\headrulewidth}{0 pt}
}

%\newcommand{\sectionbreak}{\clearpage}
%\newcommand{\subsectionbreak}{\clearpage}


%%%%%%%%%%%%%%%%%%%\renewcommand{%%%%%%%%%%%%%%%%%%%%%%%%%%%%%%%%%%%%%%%%%%%%%%%%%%%%%%%%%%%%
%%%%%%%%%%%%%%%%%%%%%%%%%%%%%%%%%%%%%%%%%%%%%%%%%%%%%%%%%%%%%%%%%%%%%%%%%%%%%%%
%%%%%%%%%%%%%%%%%%%%%%%%%%%%%%%%%%%%%%%%%%%%%%%%%%%%%%%%%%%%%%%%%%%%%%%%%%%%%%%

\begin{document}

\input{./pages/titlepage}

\pagestyle{empty}
\input{./pages/acknowledgments}
\clearpage

\pagestyle{empty}
\tableofcontents
\clearpage

\pagestyle{plain}

\input{Introduction}

\input{CompetitionConcepts}

\input{GeneralRules}

\input{Setup}

\chapter{Tests in Stage I}
\label{chap:stage_I}

\begin{itshape}
\iterm{Stage~I} comprehends five \textbf{ability tests} and an \textbf{integration test} along with an open demonstration for the audience. 
Each ability test is designed to evaluate the average performance of the robot in one particular skill, providing data for benchmarking. 
Meanwhile, the integration test has been designed to evaluate how this abilities work together while solving a common task.

The total score for ability and integration tests is the average of the best two performances out of preferably three performances (given the time constraints of a competition).
The point of this is to both eliminate good and bad luck for the robots/teams and to get a more objective view of the performance,
  not to give teams time to tweak the robot between test performances. 

\iterm{Following and Guiding} (demonstration for the audience) goes out of the arena and into the venue between the audience. 

\end{itshape}

\subsection*{Scheduling}
For maximal efficiency, teams will be scheduled interleaved: 
  Team A does an attempt while team B sets up their robot. When A is done, it moves out the way for team B, then B attempts while A sets up the robot again etc.

The preparing team should prepare their robot close to the place of the test, but not interfere with the performing robot.
Prepared robots must wait at this preparation location until commanded to start the test.
When commanded to start, the robot must move automatically beyond this point. 

Robot should be ready to start the next attempt to the same test as fast as possible: 
  when the performing robot is done with a attempt, the next robot must be ready to go with the start of a button or a voice command.

\newpage
\input{tests/CocktailParty}

\newpage
\input{tests/GPSR}

\newpage
The maximum time for this test is 5 minutes.

{\footnotesize
\begin{scorelist}
	\scoreheading{Following Phase} % 30 pts
	\scoreitem{10}{Follow operator outside the arena}
	\scoreitem{15}{Follow operator to the car}
	\scoreitem{ 5}{Understand the destination}
	
\ifNotSSPL{
	% These are mutually exclusive, max score is thus 20
	\scoreheading{Bag pick-up (OPL \& DSPL only)}
	\scoreitem{0}{Put up hand, closing it after waiting time elapses}
	\scoreitem{2}{Put up hand, closing it right after the bag is inserted}
	\scoreitem{5}{Pick up the bag from the floor}
	\scoreitem{20}{Take bag from operator's hand (autonomous handover, we expect to find the hand/bag and when the bag is released)}

	\scoreheading{DSPL \& OPL Tasks} % 80 pts
	\scoreitem{10}{Re-enter the arena}
	\scoreitem{ 5}{Deliver the bag at the specified location}
	\scoreitem{10}{Find the person at the specified location}
	\scoreitem{30}{Open door without help}
	\scoreitem{10}{Guide operator outside the arena}
	\scoreitem{15}{Guide operator to the car}
}

\ifSSPL{
	\scoreheading{SSPL only Tasks} % 100 pts
	\scoreitem{10}{Tell the time to the stranger}
	\scoreitem{30}{Re-enter the arena}
	\scoreitem{20}{Find the person at the specified room}
	\scoreitem{10}{Guide operator outside the arena}
	\scoreitem{30}{Guide operator to the car}
}
	
	\scoreheading{Obstacle avoidance} % 70 pts
	\scoreitem{20}{Avoiding small (box-sized) object}
	\scoreitem{20}{Avoiding 3D (hard-to-see) object}
\ifSSPL{%
	\scoreitem{30}{[SSPL] Asking a person to step aside (\textit{smart} obstacle)}
}%
\ifNotSSPL{%
	\scoreitem{30}{[DSPL \& OPL] Moving away movable object}
}%
	
	\setTotalScore{200}
\end{scorelist}
}

% Local Variables:
% TeX-master: "Rulebook"
% End:


\newpage
\input{tests/SPR}

\newpage
\input{tests/StoringGroceries}

\chapter{Tests in Stage II}
\label{chap:stage_II}

\begin{itshape}
All ability and integration tests in \iterm{Stage~II}  are performed only once. Some tests have optional tasks that grant additional points when performed correctly, clean and fast. The \iaterm{Technical Committee}{TC} must be informed if a team is planning to perform any of the optional tasks. Unless explicitly stated otherwise, no additional time is given while performing optional tasks.

In the \iterm{Open Challenge} the robot must be able to show to the \iaterm{Technical Committee}{TC} the achievements on the main research line of its own team. This test grants up to 250 points.

\section*{Robot \& team cooperation}
We encourage robots and teams to work together when performing tests.
For scoring, points are awarded per subtask. The robot (and thus team) performing the subtask gets the points.
For example, in the Restaurant test, if one robot of team A can take the order and another robot of team B delivers the order, then the points for taking the order go to team A, while the points for delivering go to team B. 
Of course, team A \& B can both perform the test in their own turn.

\end{itshape}

\newpage
\input{tests/EEGPSR}

\newpage
\section{Imitation}

In this test we evaluate the ability of a robot to learn from human performance
and therefore also advanced manipulation of objects like picking and placing as 
well as advanced interaction between objects like pouring. The robot should 
reason about what a person is doing with the objects it is interacting.

For this task we expect all robot behaviours from Category 1 and Category 2 of
the GPSR test. A person is executing a task, the robot should classify it by observation
and be able to execute it afterwards by translating it into known behaviours that
are performed during Stage 1 tests. Some manipulation actions also need to be
added according to the limited set of actions of Stage 1.

For that the robots needs to solve human-object relations and classify an action
based on the observation.

The task will be divided as follows:

\begin{itemize}
	\item Initial Scene Observation
	\item Action Observation
	\item Action Execution
\end{itemize}


\subsection{Focus}
This test particularly focuses on the following aspects:
\begin{itemize}
	\item No predefined order of actions to carry out.
	\item Action analysis and understanding.
	\item Increased complexity and focus on action recognition and manipulation (unlike GPSR where focus is on speech recognition and speech understanding).
	\item Environmental (high-level) reasoning.
	\item Task execution.
\end{itemize}

\subsection{Task}

\begin{enumerate}
\item \textbf{Entering:} The robot enters the apartment to a predefined location 
	and is positioned on the opposite site of a human demonstrator. After entering 
	the robot could execute an initial scene analysis. All objects that are used 
	during the performance are already placed on the table.

	\item \textbf{Action observation:} The operator starts the learning phase 
		by voice "Watch me now". The person then interacts with the objects. Potential
		actions can be: \\
		\begin{itemize}
			\item Picking
			\item Placing
			\item Pouring
			\item Stacking
			\item ...
		\end{itemize}

	\item \textbf{Action classification:} The robot should announce by voice it just analysed
		from the performance. Each task contains a sequence of three basic actions.

	\item \textbf{Action execution:} Then the robot is triggered by voice "Perform what you just saw"
		and tries to repeat the observed action. (The initial state will be recreated then (in terms of that 
		the same objects are on the table, positions and order can vary.)

\end{enumerate}

\subsection{Additional rules and remarks}
\label{sec:imitation_remarks}
\begin{enumerate}
	\item \textbf{Referees:} Since the score system in this test involves a 
		subjective evaluation of the robot's behavior, the referees are EC/TC members.

	\item \textbf{Operator:} The person operating the robot is one of the 
		referees (default operator).

	\item \textbf{Incremental scoring:} Scores will be awarded for the correct
		announcement of the classified action including the interacting objects.
		The actual execution of the correctly associated sequence of actions 
		will be awarded. Each correct sub-action will be taken into account.
\end{enumerate}

\subsection{Referee and OC instructions}
\textbf{2h before test:}
\begin{itemize}
	\item Specify objects, possible object affordances
	\item initial location.
\end{itemize}

\newpage
\subsection{Score sheet}

%TODO (Raphael): Scoresheet
%\input{scoresheets/imitation.tex}

% Local Variables:
% TeX-master: "Rulebook"
% End:


\newpage
\newcommand{\bonusRobotCoop}{50~}

\section{Open Challenge}
\label{sec:test_open_challenge}

During the Open Challenge teams are encouraged to demonstrate recent research results and the best of the robots' abilities. It focuses on the demonstration of new approaches/applications, human-robot interaction and scientific value.

\subsection{Task}

The Open Challenge consists of a demonstration and an interview part.
It is an open demonstration which means that the teams may demonstrate anything they like.
The performance of the teams is evaluated by a jury consisting of all team leaders, TC and EC.
\OpenDemonstrationTask{seven}{three}

\subsection{Presentation}
During the demonstration, the team can present the addressed problem and the demonstrated approach.
\begin{itemize}
	\item A video projector or screen, if available, may be used to present a brief (max. 2 minute) presentation relevant to the demonstration.
	\item Teams may omit the video, use a more brief video, or have the robot act over the video in order to make more time for the robot demo.
	\item There may be no human presenter. This is intended to be a demonstration of the robot's capabilities and not a research talk. The robot may present for itself (e.g., describing what it is doing or providing a narrative for the presentation on its own).
	\item Humans may interact with the robot during the interaction, but are not to act as presenters. This judgement is left to the jury.
	\item The team can also visualize robot's internals, e.g., percepts.
\end{itemize}

It is important to note that the jury may decide to end the demonstration if there is nothing happening or nothing \emph{new} is happening.

\OpenDemonstrationChanges

\subsection{Jury evaluation}
\label{sec:test_open_challenge:scoring}
\begin{enumerate}
	\item \textbf{Jury of team leaders:} All teams have to provide \emph{one} person
	(preferably the team-leader) to follow and evaluate the entire Open Challenge.
	\item \textbf{Evaluation:} Both the demonstration of the robot(s), and the answers of the team in the interview part are evaluated.\\
	For each of the following \emph{evaluation criteria}, each jury member submits a score from $0-100$:
	\begin{enumerate}
	\item Novelty and (scientific) contribution
	\item Difficulty level of the demonstrated task
	\item Success of the demonstration
	\item Overall (demo was convincing, fluent, interesting, etc.)
	\end{enumerate}
	A jury member is not allowed to evaluate and give points for the own team.
	\item \textbf{Normalization and outliers}:
	\begin{enumerate}
		\item The points given by each jury member are scaled to obtain a score from $0.0-1.0$.
		\item The normalized total score for each team is the mean of the jury member scores.
			To neglect outliers, the $N$ best and worst scores are left out:
			$$\mbox{score}_{norm} = \frac{\sum\mbox{team-leader-score}}{\mbox{number-of-teams} - (2N+1)}\times\frac{1}{100},
			\quad N=\begin{cases}2, & \mbox{number-of-teams} \ge 10\\1, & \mbox{number-of-teams} < 10 \end{cases}$$
		\end{enumerate}
		\item The final Open Challenge score for each team is computed at the end of Stage 2. The Open Challenge \scoring{final score} is the product of the normalized score multipled by the highest score achieved in Stage 2:
		$$\mbox{score} = \mbox{score}_{norm} \times \frac{min\Big(250, max\big(\{S_2\}\big)\Big)}{250},
		\quad \{S_2\}=\mbox{All Stage2 scores}
		$$
\end{enumerate}

\subsection{Additional rules and remarks}
\begin{enumerate}
	\item \textbf{Start signal:} There is no standard start-signal for this test.
	\item \textbf{Abort on request:} At any time during the demonstration, the jury may interrupt and abort the demonstration:
	\begin{enumerate}
		\item if nothing is shown: in case of longer delays (more than one minute), e.g., when the robot does not start or when it got stuck;
		\item if nothing new is shown: the demonstrated abilities were already shown in previous tests (to avoid dull demonstrations and push teams to present novel ideas).
	\end{enumerate}
\end{enumerate}

\subsubsection{Team-team-interaction:}
\label{rule:OC-team-team-interaction}
An extra bonus of up to \bonusRobotCoop points can be earned if robots from two teams (4 robots maximum, 2 from each team) successfully collaborate (robot-robot interaction).
\begin{enumerate}
	\item This bonus is earned for both teams.
	\item The robot(s) of the other team must only play a minor role in the total demonstration.
	\item It must be made clear that the demonstrations from the two teams are not similar, otherwise the points cannot be awarded.
	\item In case a team receives two (or more) bonuses, the maximum bonus will be taken.
	\item The collaboration is possible even if one of the two teams has not reached Stage 2.
	\item A team not participating in Stage 2 receives no bonus points for this test.
\end{enumerate}

\paragraph*{Inter-league collaboration}:
\label{rule:OC-inter-league-collaboration}
Inter-league collaboration must be announced to the OC at least one day before the test. Teams participating in multiple @Home Leagues does receive no bonus for cooperation. Standard Platform robots are allowed to take part in the Open Challenge of the Open Platform League, but Open Platform robots can \emph{not} participate in any Standard Platform League's test. In the same sense, DSPL robots are not allowed in SSPL and vice versa.

For sake of clarity, please consider the following example: Let be A, B two teams participating in RoboCup @Home where
\begin{itemize}
	\item Team A participates in SSPL.
	\item Team B participates in both SSPL and OPL.
	\item Team A and B have qualified into Stage II.
\end{itemize}
Then, by applying the \textit{Inter-league collaboration Rule} (See \refsec{rule:OC-inter-league-collaboration}) the following statements can be concluded:
\begin{itemize}
	\item B OPL can not participate in A SSPL's open challenge.
	\item B OPL can not participate in B SSPL's open challenge.
	\item A SSPL can participate in B OPL's open challenge. Team A and B get a bonus because A <> B.
	\item B SSPL can participate in B OPL's open challenge. There is no bonus because B = B.
\end{itemize}




% Local Variables:
% TeX-master: "Rulebook"
% End:


\newpage
\input{tests/Restaurant}

\newpage
\input{tests/SetATableTidyUp}

\newpage
The maximum time for this test is \textbf{10 minutes}.

\begin{scorelist}
	\scoreheading{Engaging spectators} % Max 50
	\scoreitem{30}{Find an spectator (or group)}
	\scoreitem{20}{Greet an spectator (handshake)}
	\scoreitem{10}{Greet and get greet by an spectator (bowing or waving)}

	\scoreheading{Guiding spectators} % Max 50
	\scoreitem{10}{Convince spectator to follow}
	\scoreitem{40}{Reach the audience area}

	\scoreheading{Q\&A Session} % Max 210
	% \scoreitem{10}{Finish talk without loosing spectators attention}
	\scoreitem{10}{Finish talk without loosing spectators}
	\scoreitem[2]{70}{Each correctly understood question}
	\scoreitem[2]{30}{Each correctly answered question}
	
	\scoreheading{Bilingual interaction} % Max 80
	\scoreitem{10}{Bilingual engaging}
	\scoreitem[2]{25}{Questions in $2^{nd}$ language}
	\scoreitem[2]{10}{Question answered also in $2^{nd}$ language}
	
	\setTotalScore{390}
\end{scorelist}


% Local Variables:
% TeX-master: "Rulebook"
% End:


\newpage
\chapter{Finals}

The competition ends with the Finals on the last day, where the four teams with the highest total score compete.
The \iterm{Finals} are conducted as a final open demonstration.
This demonstration does not have to be different from the Open Challenge. 
It does not have to be the same either.

To avoid logistical issues during the last day of the competition, the \iterm{Finals} are divided into two sets of demonstrations: the Bronze Competition and the RoboCup @Home Grand Finale.
The Bronze Competition is a set of demonstrations that are carried out before the RoboCup @home Grand Finale. Here, all the leagues run in parallel, with the fourth and third highest scored teams competing for the bronze.
Finally, the two teams with the highest score in each League present their demonstrations in a serialized manner during the RoboCup @Home Grand Finale.

Even though each league has its own first, second and third place, the RoboCup @Home Grand Finale is meant to show the best of all leagues to the jury members as well as the audience and, thus, warrants a single schedule slot.

\section{Evaluating Juries for Final Demonstrations}
\label{final:jury}
Each set of final demonstrations is evaluated by a different combination of evaluating juries, here described.

\begin{enumerate}
\item\textbf{League-internal jury:} The league-internal jury is formed by the Executive Committee.
The evaluation of the league-internal jury is based on the following criteria:
  \begin{compactenum}
  \item Scientific contribution %(maybe taken from the OC)
  \item Contribution to @Home %(evaluated by Execs/TC)
  \item Relevance for @Home / Novelty of approaches %(evaluated by execs/TC)
  \item Presentation and performance in the finals.
  \end{compactenum}

\item \textbf{League-external jury:} The league-external jury consists of people not being involved in the RoboCup@Home league,
but having a related background (not necessarily robotics).
They are appointed by the Executive Committee.
The evaluation of the league-external jury is based on the following criteria:
  \begin{compactenum}
  \item Originality and Presentation
    (story-telling is to be rewarded)
  \item Usability / Human-robot interaction
  \item Multi-modality / System integration
  \item Difficulty and success of the performance
  \item Relevance / Usefulness for daily life
  \end{compactenum}

\item\textbf{Teams-based jury:} The teams-based jury is formed by members of the league's teams.
The evaluation of the teams-based jury is based on the following criteria:
  \begin{compactenum}
  \item Scientific contribution %(maybe taken from the OC)
  \item Contribution to @Home %(evaluated by Execs/TC)
  \item Relevance for @Home / Novelty of approaches %(evaluated by execs/TC)
  \item Presentation and performance in the finals.
  \end{compactenum}
\end{enumerate}


\section{Bronze Competition (4th and 3rd Highest Scoring Teams)}
The demonstration is evaluated by one member of the league-internal jury, by one member of the league-external jury and by the complete team-based jury.
The final score and ranking are determined by the jury evaluations and by the previous performance (in Stages I and II) of the team, in the following manner:

\begin{enumerate}
  \item The influence of the league-internal jury member to the final ranking is \SI{15}{\percent}.
  \item The influence of the league-external jury member to the final ranking is \SI{15}{\percent}.
  \item The influence of the teams-based jury to the final ranking is \SI{15}{\percent}.
  \item The influence of the total sum of points scored by the team in Stage I and II is \SI{55}{\percent}.
\end{enumerate}

These demonstrations are carried out in parallel, having each League perform their own Bronze Competition in their own arena at the same time to save time.

\section{RoboCup@Home Grand Finale (2nd and 1st Highest Scoring Teams)}
The demonstration is evaluated by the complete league-internal and the complete league-external jury.
The final score and ranking are determined by the jury evaluations and by the previous performance (in Stages I and II) of the team, in the following manner:
  
\begin{enumerate}
  \item The influence of the league-internal jury to the final ranking is \SI{25}{\percent}.
  \item The influence of the league-external jury to the final ranking is \SI{25}{\percent}.
  \item The influence of the total sum of points scored by the team in Stage I and II is \SI{50}{\percent}.
\end{enumerate}

These demonstrations are carried out in a serialized fashion, one League performing after another in one arena.


\section{Common Description of Final Demonstrations}
Teams can choose freely what to demonstrate, however it is expected that teams present the scientific and technical contributions they submitted in both \iterm{team description paper} and the \iterm{RoboCup\char64Home Wiki}.
In addition, teams may provide a printed document to the jury (max 2 pages) that summarizes the demonstrated robot capabilities and contributions.  

\subsection{Task}
The procedure for the demonstration and the timing of slots is as follows:
\OpenDemonstrationTask{ten}{five}

\OpenDemonstrationChanges

%% %%%%%%%%%%%%%%%%%%%%%%%%
\section{Final Ranking and Winner}

The winner of the competition is the team that gets the highest
ranking in the finals.

There will be an award for 1st, 2nd and 3rd place. All teams in the
Finals receive a certificate stating that they made it into the Finals
of the RoboCup@Home competition.


% Local Variables:
% TeX-master: "Rulebook"
% End:


\input{Appendices}

\printabx
\printidx

\end{document}
